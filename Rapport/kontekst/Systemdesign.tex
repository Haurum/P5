\chapter{System design}
\label{chap:System design}
This chapter will describe some of the software choices made, as well as the robot design. The program design, the robots positioning and movement and how the Arduino and Kinect are connected, will also be described in this chapter. 


\section{Software choices}
\label{sec:Software choices}
In this section the software choices made for the Arduino IDE is described. Also, the libraries being used for making the program will be described as well. 


\subsection{Arduino IDE}
\label{sec:Arduino IDE}
While testing the Arduino IDE 1.6.12 with the code listed on the Arduino website’s wifi-guide \citep{wg}, it is discovered that the wifi library is not compatible in newer versions of the IDE. The library is supported in version 1.0.3, which is why this particular IDE version is used in this project. The newer IDEs have increased security, which would actively refuse any TCP connection from the computer.


\subsection{Libraries}
\label{sec:Libraries} 
To make use of the Motor and Wifi Shields, it is decided that two libraries are included in the project, although similar functionality could have been written by hand instead. Constructing this functionality is not a learning goal for this project and therefore the libraries are included. 


\subsubsection{Motor.h}
\label{sec:Motor.h}
The Motor.h library is being used for the project, making it possible to control the motors independently of each other, e.g. the motors can be set to different speeds and can be stopped at different times. 


\subsubsection{Wifi.h}
\label{sec:Wifi.h}
Arduino wifi library is used to create the connection between the Wifi Shield and the router. This library has methods for creating a server, where clients can both read from and write to the server. The subsection \ref{sec:Arduino IDE} explains complications between the versions of the Arduino IDE, and why the Arduino IDE 1.0.3 is used for this project.


\section{Robot design}
\label{sec:Robot design}
The purpose of the robot is to catch the thrown object, by driving to the impact point of the object and the ground within the predefined area. The final design of the robot is shown in figure \ref{robot}.


\begin{figure}[h]
	\centering
	\fbox{\includegraphics[scale=0.20]{billeder/robot}}
	\caption{The final design of the robot}
	\label{robot}
\end{figure}


The robot uses two LEGO NXT Servo motors with compatible wheels, which will act as the front wheels. The back wheel can slide from side to side, which is important, since the wheel will be dragged sideways when the robots turns, whereas a normal rubber wheel would have caused a lot of friction. 
The robot is build around the NXT Servo motors, with a platform at the top. The Arduino is strapped to the platform, with the shields on top of it. This ensures that the Arduino does not slide off the robot while moving.
The robot is build with the arduino at the top, for easy access to the different pins if needed. 


\section{Arduino program design}
\label{sec:Arduino program design}
As mentioned in section \ref{sec:Software choices}, the arduino program is developed in the Arduino IDE version 1.0.3. The Arduino code controls the behaviour of the robot, which calculates the movement, keeps track of positioning, and reacts on input from the Kinect sensor. This program is responsible for the intercommunication between separate parts of the robot.


The program has two predefined Arduino functions called setup and loop. The setup function will be executed once when the program is started, and the loop function will, as the name indicates, loop over and over until the program is completed or exited. The loop function is the behaviour of the robot. \newline 
The setup function initializes the wifi connection to the computer running the Kinect software, which is used to receive predicted coordinates from the Kinect. After connecting the Arduino and the computer, the program will wait for the first set of coordinates. When the coordinates have been received, the program will execute the behaviour according to those coordinates. The robot will move towards the coordinates, and keep track of its own position. It will check whether or not it has hit the impact point or not, and when its own position is the same as the impact point it will exit the program. The behaviour of constantly updating the position and comparing that to the impact point, and moving towards the coordinate, will loop until it has exited. The initialization and behaviour of the robot can be expressed as the following tasks:


\begin{itemize}
	\item Make connection to the Kinect
	\item Wait to receive coordinates from the Kinect
	\item It will keep waiting for the Kinect to send a set of coordinates, when the coordinates is received, it  will loop over the behaviour of the robot:
	\begin{itemize}
		\item Check if already at the collision point, if it is exit program
		\item Start moving, adjust direction relative to heading
		\item Update its position
	\end{itemize}
\end{itemize}
The flow of the tasks are further described with a flowgraph shown in figure \ref{figure:flowgraph}.

\begin{figure}[h]
	\centering
	\includegraphics[scale=0.25]{billeder/flowgraph.png}
	\caption{The flowgraph of the program}
	\label{figure:flowgraph}
\end{figure}


\section{Robot positioning and movement}
\label{sec:Robot positioning and movement}
The idea for moving the robot is to send coordinates to the robot, which are converted to explicit millimetre values of the distance from the starting point (0, 0). This starting point delimits a predefined area, as described in Appendix \ref{sec:i1Predefined areaImplementation}, which is the area in which the robot should catch the thrown object. This area is defined by the robot’s capabilities and limitation. This area would allow the robot to calculate the distance travelled, and compare that value to the distance of the received coordinate. The NXT Servo motor encoders should be used to calculate the distance travelled. \newline
If the robot has read a coordinate and started to move towards it, receiving a new coordinate would redirect the robot to the new coordinate instead. As described in Appendix \ref{sec:i1MovementImplementation}, the robot’s maximum speed was 25.5 cm/s, but it is significantly slower moving backwards rather than forward. This resulted in the definition of the robot always starting at the coordinate (0, 0), and not considering any trash thrown behind the robot. The predefined area is determined to only be in front of the robot, to ensure that the robot doesn't move backwards. 


\section{Connecting Arduino and Kinect}
\label{sec:Connecting Arduino and Kinect}
For convenience, and as a part of the requirements for this project, the Arduino should receive wireless data from a computer, connected to the Kinect. The Arduino Wifi Shield makes this possible, with the Wifi library included in the Arduino program. \newline
The connection between the Arduino and the computer should be through a local router, with a set SSID and password. The computer should use the Kinect to calculate an impact point of the object, and send this in a specific format, so that the data can be easily read and translated to a set of coordinates for the robot’s movement. The computer should be able to send coordinates more than once, since the first impact point is not necessarily precise enough to catch the object. The computer should calculate increasingly accurate impact points, which should be sent with a certain minimal inter-arrival time, to not hinder the robot’s movement, and yet still in time for the robot to correct itself. \newline
The Arduino should receive this data and read whenever it has sufficient time to do so, and convert the data received to match the right coordinate in the predefined area. 


\section{Scheduling}
\label{sec:Scheduling}
The subsection, Cyclic Executive, details the scheduling model used in this project. The tasks, and the model type, will be explained, and further analysed in the subsection, Real-Time analysis, to express information about how the scheduling was achieved.
The subsections, Interrupts and their definitions, Stack Overflow and Interrupt Overload use the article "Safe and Structured Use of Interrupts in Real-Time and
Embedded Software" written by John Regehr \citep{safe}.


\subsection{Cyclic Executive}
\label{sec:Cyclic Executive}
A cyclic executive is a model which assumes a fixed set of periodic tasks. The model is about designing an entirely static schedule in which the tasks are cyclically executed at their rate, since they are periodic, and must also meet their deadline. The model is cyclic since when the last task ends and the allotted time of a cycle ends, a new cycle begins, with all tasks placed in the same specific order as the previous cycle. The creation of such a schedule proves by construction that the tasks will always meet their deadlines at runtime, if the assumptions of the model are true. \newline
To ensure that the cyclic executive is in synchronization with real elapsed time, synchronization points are inserted into the code that implements it.
Once cyclic executives are constructed, the implementation is simple and efficient since there is little overhead as there is no scheduling at runtime. Cyclic executives can be constructed separately from the system, by hand or by a tool. \citep{rtsbog}


\subsection{Interrupts and their definition}
\label{sec:Interrupts and their definition}
When describing interrupts, one must first define what interrupts are. The definitions are quotes from John Regehr's article. The definition for an interrupt is twofold, as the first part is as follows:


{\addtolength{\leftskip}{10 mm}
	\enquote{A hardware-supported asynchronous transfer of control to an interrupt vector based on the signaling of some condition external to the processor core}
	
}


The second part of the definition is:


{\addtolength{\leftskip}{10 mm}
	\enquote{An interrupt is the execution of an interrupt handler: code that is reachable from an interrupt vector}
	
}


The definition for the introduced term interrupt vector:


{\addtolength{\leftskip}{10 mm}
	\enquote{A dedicated or configurable location in memory that specifies the address to which execution should jump when an interrupt occurs}
	
}


Interrupts don’t always return control flow to where the interrupt disrupted control flow. Interrupts often change the state of main memory, and that of device drivers, but does not disturb the main processor context of the computation which was disturbed. \newline
An interrupt is pending when it's firing condition has occurred, the interrupt controller has been updated and the interrupt handler has not started executing. A missed interrupt is when the firing condition occurs but it does not become pending, this is usually because another interrupt of the same type is already pending. Most hardware platforms, including the atmega2560, use a bit to distinguish whether an interrupt is pending or not, which means it does not track if there is more than one interrupt pending of that type. Hardware support can be used to disable interrupts by manipulating bits in hardware registers, either through the master interrupt enable bit, or the enable bits for each interrupt. 


The following conditions decide when the firing condition for an interrupt is true:


\begin{itemize}
	\item The interrupt is pending
	\item The processors master interrupt enable bit is set
	\item The enable bit for each interrupt bit is set
	\item The processor is in between executing instructions 
	\item There exists no higher priority interrupt which fulfills 1-4
\end{itemize}


Another important part of interrupts is interrupt latency, which is the time between the interrupt firing conditions become true, and the first instruction of the interrupt handler has begun executing. Nested interrupts is when an interrupt handler is preempted by another interrupt. An important distinction between threads and interrupts is that thread scheduling is through software, while interrupt scheduling is through hardware interrupt controller.


\subsection{Stack Overflow}
\label{sec:Stack Overflow}
A stack overflow is when a stack grows beyond the confines of the memory allocated to it, thereby corrupting RAM and could cause system malfunction and crash. Allocating memory to a stack is about a balance between enough memory to prevent stack overflow from occurring and not assigning more memory than needed. There are two ways to approach this problem, analysis and testing based. A significant advantage of analysis is that it can be performed quickly through tools, in contrast to testing which is much more laborious. \newline 
The testing approach is about running the system and observing how large the stack grows. This approach is a kind of black box testing, it doesn’t matter how or why memory is used, all that matters is how big the stacks becomes. A problem with the testing based approach is that it can miss some of the program paths, where it might have taken a branch at some point during the execution of the program which causes greater stack growth than the paths tested. \newline
The analysis based approach is concerned with the control flow and its goal is to find the path which pushes the most data onto the stack. The problem with the analysis based approach is that it's often too optimistic about the maximum stack size to prevent stack overflow, since it accomadates behaviour which occur very rarely or never. \newline
Stack overflow is a general problem in embedded systems, but was not tested for this project, this is addressed as part of \ref{chap:FutureWork}. 

\subsection{Interrupt overload}
\label{sec:Interrupt overload}
Interrupt overload is when interrupts occur so frequently they dominate the processor from performing its other computations, which in the worst case can end up starving other important processes. While it may be intuitive to think that large interrupt payloads are a natural cause of interrupt overload, but it is rather unexpectedly large interrupt loads which can cause interrupt overload. A reliable maximum interrupt request rate is a important thing to ensure that it is accurately evaluated to have a reliable real-time system.


To discover if interrupt overload can occur, it is possible to analyze how much cpu utilization can be spent handling interrupts, the maximum time spent in an interrupt handler can be calculated by the maximum execution time of a given interrupt multiplied by the worst-case arrival rate. \newline
The maximum execution time of the interrupt handler in our system is 40 clock cycles, 10 of which are used to enter and exit the interrupt handler with the remaing 30 being the actual interrupt handler code, and the worst-case arrival rate is 1 occurrence per millisecond. The worst-case arrival rate is based on the maximum speed of the NXT servo motor and the motor encoder which reads it, which causes interrupts to be generated. \newline
The maximum time spent in an interrupt handler in our system is \begin{math} \frac{40 \cdot 1}{16000} = 0.0025\% \end{math} of cpu spent dealing with interrupts, where the 16.000 comes from the cpu clock cycles per millisecond. The interrupts in this project listens to the motor encoders for change, each time a change happens(the change is between high and low voltage), it will interrupt the program and in this project it will increment a counter. Since only one type of interrupt with a relatively low amount of cpu utilization is used, interrupt overload cannot occur for this system. 


\subsubsection{Real-Time analysis}
\label{sec:Real-Time analysis}
The previously addressed problems of stack overflow and interrupt overload are subproblems of the real-time schedulability analysis, especially concerning interrupts, which will make sure that all computations will be met within their time constraints. \newline
To schedule the system a cyclic executive is used, which is mentioned in section \ref{sec:Cyclic Executive}, which handles many of the problems addressed, as long as its assumptions are true. 
Since the cyclic executive has predetermined behaviour it will have no jitter for the execution of each of the periodic tasks. Although jitter has not been a particular concern in this project.\newline Jitter is the concept of time variation in a computed output being messaged to external environment from period to period. \newline
The strengths of using a cyclic executive include: execution schedule is predetermined, simple, efficient and fast, no jitter, prevents race-conditions and deadlock. 
The existence of a cyclic executive is a proof by construction that the real-time analysis will hold, and the problems which arise from interrupts have also been addressed in this chapter.
