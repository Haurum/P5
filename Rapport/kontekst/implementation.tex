\chapter{Implementation}
\label{chap:Implementation}
This chapter will describe the implementation of the systems components. It will cover the description of the Microsoft Kinect, the robot's code and how the scheduling is done. 

\section{Microsoft Kinect}
\label{sec:Microsoft Kinect Implementation}

\section{Robot}
\label{sec:Robot}

\section{Scheduling}
\label{sec:Scheduling implementation}
To calculate the worst-case execution time(WCET) for the systems tasks, several methods were tried out, which is described in Appendix \ref{chap:Increment three} in section \ref{sec:i3Scheduling}. 
The method used in this project to get the WCET for the system tasks, is to use the micros() function in the arduino IDE. \newline
Each task running time is measured with the micros() function, by running the function several time at using the highest time value as WCET. This was done for the functions updatePosAndHead(), driveTowardsGoal() and for when the Kinect program sends data till the arduino have received it via WiFi.

The results of the WCET of the tasks using the micros() function:
\begin{itemize}
	\item updatePosAndHead = 1076 microseconds
	\item driveTowardsGoal = 732 microseconds
	\item WiFi = 8433 microseconds
\end{itemize}
These results was used to make a schedulability analysis in the UPPAAL program, which will be described in the following section.

\subsection{UPPAAL schedulability analysis}
\label{sec:UPPAAL schedulability}
As described in Appendix \ref{sec:i3UPPAAL model}, Dumpsty contains three tasks: PrA, PrB and PrC. These three tasks all have individual WCET, which is not calculated, but rather tested, since calculating these through assembly proved to be impossible due to unbound loops in libraries. After testing the individual task's WCET, the worst case found would be significantly faster than what is labelled in UPPAAL, since the probability of hitting the actual worst case is close to impossible with the amount of tests done. In the following bulletpoints, all three tasks tested WCET and the WCET used in UPPAAL for the specific task is expressed, which is the first step to verify the schedulability of Dumpsty's tasks.

\begin{itemize}
	\item PrA \tab Tested: 1067 microseconds \tab UPPAAL: 2 milliseconds
	\item PrB \tab Tested: 732  microseconds \tab UPPAAL: 1 milliseconds
	\item PrC \tab	Tested: 8469 microseconds \tab UPPAAL: 9 milliseconds
\end{itemize}

For convenience, and to simplify the analysis, all tasks contain the worst case runtime of all interrupts and interrupt handlers that might occur during the execution of the task. These interrupts are generated by the motor encoders when Dumpsty is moving.

Figure 5.1 depicts the automatas created in UPPAAL from two declared classes. The first class, task PrA, is the cyclic executive instance for the task PrA. The second class is simply a CPU which is the key needed to run the task. Every task can grab the CPU, but only one may hold it at any given time. The CPU is then released when the task is done executing, and another task can then proceed to run.

\begin{figure}[h]
	\centering
	\includegraphics[scale=1]{billeder/UPPAALPr}
	\caption{Automata in UPPAAL}
	\label{robot}
\end{figure}
