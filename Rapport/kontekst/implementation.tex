\chapter{Implementation}
\label{chap:Implementation}
This chapter will describe the implementation of the systems components. It will cover the description of the Microsoft Kinect, the robot's code and how the scheduling is done. 

\section{Microsoft Kinect}
\label{sec:Microsoft Kinect Implementation}
This section will describe how the Kinect sends coordinates to the Arduino.
The Kinect gives information about the balls predicted impact point, in relation to its field of view and the distance to the point. The coordinate has to be converted to fit in a coordinate system that has it’s 0,0 at the Kinect, X-axis out from the side of the Kinect, and Y-axis out perpendicular from the front of the Kinect.
An detailed description of the conversion can be found in \ref{sec:i3Connecting Arduino and Kinect implementation}.

The coordinate can then be send to the Arduino, using its IP-address and a tcpClient. 
When writing to the tcpClient’s stream, the information is send to the server on the Arduino, that should be waiting to retrieve it.
The code for this can be seen here:

\begin{lstlisting}[caption={Sending data to the Arduino}, label={sendData}]
tcpClient = new TcpClient();
tcpClient.Connect("192.168.0.100", 9999);
stream = tcpClient.GetStream();

stream.Write(Encoding.UTF8.GetBytes((finalx.ToString() + ";" + finaly.ToString()).ToCharArray()), 0, (finalx.ToString() + ";" + finaly.ToString()).Length);
tcpClient.Close();
\end{lstlisting}


\section{Robot}
\label{sec:Robot}
This section will describe arduino code for the robot. This will include how the arduino connects to the Kinect and the robots behaviour and calculations done by the arduino. If there was any comments in the original code, they have been removed in the listings for this chapter. \newline
A problem occurred when both the WiFi shield and the motor shield should work together. When the WiFi shield is trying to connect to the network and if any of the motor shield components are initialized at the beginning of the program, it isn't possible to connect to the network. If it is connected to the network before the motor shield components are initialized, it is possible to receive data and use the motor shield components.

The arduino setup function, is a function run once at the beginning of the running program, to setup the program as the name implies. In listing \ref{lst:setup} the setup function is shown, which in this project will connect to the private network, using the guide from Arduinos own website, and get the first coordinate \citep{wg}. \newline
It will try to connect to the specific network by passing the network name and password to the function WiFi.begin(), given by the WiFi library. If the arduino can't connect to the network, the program will exit. If it successfully connects, the arduino will call the function server.begin() to initialize a serve that can receive data, and is now connected to the network.
The \textbf{dataString}  variable contains the data received, so the while loop makes sure that the program waits for the coordinates before continuing. A client variable will be initialized using server.available(), which will make the arduino ready to receive data. The function client.available(), returns true if the server has received any information. If true, the data is appended to the dataString, and execution of the program will continue.
\newline
goalX and goalY is then extracted from the \textbf{dataString} and converted to integers.

\begin{lstlisting}[caption={The arduino setup function}, label={lst:setup}]
void setup() {
	Serial.begin(9600);           
	status = WiFi.begin(ssid, pass);
	if ( status != WL_CONNECTED) { 
		exit(0);
	} 
	else {
		server.begin();
	}
	ip = WiFi.localIP();

	while(dataString == ""){
		client = server.available();
		if (client.connected()){
			while (client.available()) {
				char c = client.read(); 
				dataString += c;
			}
		}
	}
	goalX = dataString.substring(0,dataString.indexOf(';')).toInt();
	goalY = dataString.substring(dataString.indexOf(';')+1).toInt();  
	
	pinMode(LEFTENCODERPIN, INPUT);
	attachInterrupt(3, incrementLeft, CHANGE);
	pinMode(RIGHTENCODERPIN, INPUT);
	attachInterrupt(2, incrementRight, CHANGE);
	
	leftTemp = leftTotal;
	rightTemp = rightTotal;
}
\end{lstlisting}



Now that the robot knows it destination it needs to calculate how to get there, but before it does that it needs to know its own position and heading. The system used to determine position and heading is defined as follows:

Position is defined as a x and y coordinate like so: (x,y). \newline
Heading 0 is straight up parallel to the y axis \newline
Headings clockwise from the y axis are negative and headings counterclockwise are positive. \newline
Heading can range from -179deg to 180deg. \newline

The robots motors each has an encoder that will give a high volt signal to the arduino for every 2 degrees the motor turns. The arduino is set up to interrupt on pin change, which means the interrupt handler for each motor will be executed for each degree the motor turns. Each interrupt handler increments an int, leftTotal for the left motor and rightTotal for the right motor. \newline
Using these two variables, postition and heading of the robot can be calculated, or rather the delta postion and heading compared to the last time position and heading was calculated.
First the values of leftTotal and rightTotal are saved to two new variables currentLeft and currentRight such that they dont change while using them in calculation. Now to find how far the robot has moved since last calculation, the last iterations degree count called temp(Left/Right) is subtracted from the current to get the delta degree count. To get the actual distance each wheel has moved, the delta degree count is multiplied by the amount of millimeters the wheel travels per degree. The position of the robot is set to be between the wheels, so to get how far the position has moved the two distances that the wheels have moved are added together and devided by 2. Next, to find the new position in coordinates the distance traveled is multiplied with sinus to the heading to get the change in the x direction and cosinus to the heading to get the change in the y direction. These two values are added to the current position to get the new position. Now to get the change in heading, the distance the right wheel traveled is subtracted by the distance the left wheel traveled to get the distance the 2 traveled compared to each other. Arctan to this value devided by the distance between the wheels gives the change in heading, which is added to the current heading to get the new one.
\begin{lstlisting}[caption={updatePosAndHead function}, label={lst:upah}]
void updatePosAndHead(){
	int currentLeft = leftTotal;
	int currentRight = rightTotal;
	double deltaLeft = (currentLeft - leftTemp) * DISTPRDEGREE;
	double deltaRight = (currentRight - rightTemp) * DISTPRDEGREE;
	leftTemp = currentLeft;
	rightTemp = currentRight;
	double dist = (deltaLeft + deltaRight) / 2.0;
	posX += (dist * sin(heading));
	posY += (dist * cos(heading));
	heading += (atan((deltaRight - deltaLeft) / WHEELDIST));
}
\end{lstlisting}

The robot now knows its destination plus its own position and heading. To get to the destination the robot can do 4 things: drive forward on one wheel and let the other stand still or vice versa, drive both wheels forward or stand still on both wheels. So basically left, right, forward or nothing. The arduino first determines if the robot is already at the goal position or it needs to move. To do this it simply compares its current position with the final destination and determines if this difference is within a give margin of error. If it is, it stops and exits the program.

\begin{lstlisting}[caption={If-statement to check if the robot should stop}, label={lst:stopplz}]
if(posX <= goalX + margin && posX >= goalX - margin && posY <= goalY + margin && posY >= goalY - margin){
	motorLeft.run(RELEASE);
	motorRight.run(RELEASE);
	delay(50);
	exit(0);
} 
\end{lstlisting}

If it is not the robot needs to be moved. To figure out whether to go left, right or forward the arduino calculates the goal heading that is needed at the current postition to point directly (or almost directly) at the goal destination. Next this heading is compared to the current heading and the arduino determines whether a right, left or no turn is needed. This is done like this:
First the goal position is subtracted by the current position to get how far the robot needs to travel in the x and y direction, from here called deltaX and deltaY. \newline
If deltaX is 0, the goal is either straight up parallel to the y axis at heading 0 or straight down parallel to the y axis at heading 180 or PI in radians. \newline
If deltaX is bigger than 0, arctan is taken to deltaY devided by deltaX, this gives the angle to the positive x axis. This is subtracted by 90 degrees (PI/2 in radians) to get the angle to the positive y axis, which is the goal heading. \newline
If deltaX is smaller than 0, almost the same thing as above happens. The only difference is that the angle from the arctan calculation will give en angle to the negative x axis, so the results is added to 90 degrees (PI/2 in radians) instead of substracted by it.

\begin{lstlisting}[caption={Part of the driveTowardsGoal function to calculate the goalHeading}, label={lst:calcGH}]
double deltaX = goalX - posX;
double deltaY = goalY - posY;
double actualGoalHeading;
if(deltaX == 0 && deltaY > 0)
	actualGoalHeading = 0;
else if(deltaX == 0 && deltaY < 0)
	actualGoalHeading = PI;
else
	actualGoalHeading = deltaX >= 0 ? atan(deltaY/deltaX) - (PI/2) : atan(deltaY/deltaX) + (PI/2);
\end{lstlisting}


The difference between the current heading and goal heading is then calculated by subctracting the current heading from the goal heading, this is from here on called the deltaHeading. \newline
If deltaHeading is larger than 180 degrees (PI in radians) it whould be faster to go the other way around, so deltaHeading is subtracted by 360 degrees (2PI in radians). \newline
if deltaHeading is smaller than -180 degrees (-PI in radians) the same thing aplies, but the other way so 360 degrees (2PI in radians) is added instead. \newline
Lastly it is determined if the deltaheading calls for a left turn, a right turn or full speed ahead.

\begin{lstlisting}[caption={Check the difference of the actualGoalHeading and its current heading, and determines which motors to run}, label={lst:heading}]
double deltaHeading = actualGoalHeading - heading;
if(deltaHeading > PI)
	deltaHeading -= 2*PI;
else if(deltaHeading < -PI)
	deltaHeading += 2*PI;
if(deltaHeading < -0.1){
	motorLeft.run(FORWARD);
	motorRight.run(RELEASE);
}else if(deltaHeading > 0.1){
	motorLeft.run(RELEASE);
	motorRight.run(FORWARD);
}else{
	motorLeft.run(FORWARD);
	motorRight.run(FORWARD);
}
\end{lstlisting}

\section{Scheduling}
\label{sec:Scheduling implementation}
To be able to schedule the systems functions(also called tasks), the worst-case execution time(WCET) have to be known for the individual functions. The first attempt was to count the clock cycles in the assembly file of the complied arduino code. It was known that the Arduino mega 2560 has 16000 clock cycles per millisecond, so it will be possible to calculate the time for the functions. The purpose was to count the clock cycles for the two functions driveTowardsGoal() and updatePosAndHead(), to make that a possibility many of the functions in the different libraries used also had to be counted, the results of this can be found in Appendix \ref{chap:Clock cycles}.

In the following list, the clock cycles counted of the two functions are shown:
\begin{itemize}
	\item driveTowardsGoal = \textbf{4185} + (13)* + (174 + (13)*)* + (231 + (13)*)* + (17)* + (8)* + (33 + (17)*)* + (7)* + (5)* + (290 + (5)*)*
	\item updatePosAndHead = \textbf{3560} + (11)* + (13)* + (7)* + (174 + (13)*)* + (231 + (13)*)* (10)* + (24)* + (17)* + (33 + (17)*)*
\end{itemize}
The bold number is the worst-case of clock cycles counted in the function. The numbers in parentheses are loops, which bound is unknown, therefore it is not possible to make a count the exact WCET by counting the clock cycles.

Since it was not possible to count the clock cycles from the assembly code, it was decided to use the tool Bound-T, which will calculate the WCET outputted with an output in clock cycles. \newline
Bound-T was not able to calculate the WCET for the code. When Bound-T gets to calculating the floats, it is failing. It can't set an upper bound for the floating point numbers' WCET.

To be able to work around the float issues with Bound-T, a library called AVRFIX made by Maximilian Rosenblattl and Andreas Wolf \citep{AVRFIX}. In the listings \ref{Update1} and \ref{Update2} a function written with the AVRFIX library and a function written normally for arduino can be compared. Be aware that the function in listing \ref{Update2} might not look exactly like this in the final code for the project. 

\begin{lstlisting}[caption={The function updatePosAndHead with AWRFIX library}, label={Update1}]
void updatePosAndHead(){
	int currentLeft = leftTotal;
	int currentRight = rightTotal;
	fix_t distPrDeg = ftok(DISTPRDEGREE);
	fix_t dltL = itok(currentLeft - leftTemp);
	fix_t dltR = itok(currentRight - rightTemp);
	fix_t deltaLeft = mulk(dltL, distPrDeg);
	fix_t deltaRight = mulk(dltR, distPrDeg);
	leftTemp = currentLeft;
	rightTemp = currentRight;
	fix_t deltaSum = deltaLeft + deltaRight;
	fix_t dist = divk(deltaSum,ftok(2.0));
	fix_t sinHeading = sink(heading);
	posX += mulk(dist, sinHeading);
	fix_t cosHeading = cosk(heading);
	posY += mulk(dist, cosHeading);
	fix_t rel = divk((deltaRight - deltaLeft),ftok(WHEELDIST));
	heading += atank(rel);
}
\end{lstlisting}

\begin{lstlisting}[caption={The function updatePosAndHead from the Arduino IDE}, label={Update2}]
void updatePosAndHead(){
	int currentLeft = leftTotal;
	int currentRight = rightTotal;
	double deltaLeft = (currentLeft - leftTemp) * DISTPRDEGREE;
	double deltaRight = (currentRight - rightTemp) * DISTPRDEGREE;
	leftTemp = currentLeft;
	rightTemp = currentRight;
	double dist = (deltaLeft + deltaRight) / 2.0;
	posX += (dist * sin(heading));
	posY += (dist * cos(heading));
	heading += (atan((deltaRight - deltaLeft) / WHEELDIST));
}
\end{lstlisting}

The program was rewritten with the AVRFIX library and then using Bound-T again to calculate the worst-case execution time for the functions. The AVRFIX library came with a .txt file from the GitHub download, which included the clock cycles for the AVRFIX functions. \newline
The function updatePosAndHead() is calculated to use 3995 clock cycles, but when trying to calculate the function driveTowardsGoal() another problem occurred. When a division by 0 took place, the program will enter an infinite loop, making it impossible to set an upper bound on the WCET. With this problem it is not possible to calculate the WCET for driveTowardsPoint().

Because using the AVRFIX library with Bound-T also failed, the group decided to use the function micros() on the functions to measure the time spent on the function, run them several times and use highest result as WCET. Besides the functions updatePosAndHead() and dirveTowardsGoal() the WCET, when the Kinect program sends data till the arduino received it via the WiFi, was also measured. 

The WCET for the two functions and the time spent on sending data from the Kinect via wifi, using the Micros() function, is listed below:
\begin{itemize}
	\item updatePosAndHead = 1076 microseconds
	\item driveTowardsGoal = 732 microseconds
	\item WiFi = 8433 microseconds
\end{itemize}

These results will be used to make a schedulability analysis in the UPPAAL program, which will be described in the following chapter.
