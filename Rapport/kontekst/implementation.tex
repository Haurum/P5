\chapter{Implementation}
\label{chap:Implementation}
This chapter will describe the implementation of the systems components. It will cover the description of the Microsoft Kinect, the robot's code and how the scheduling is done. 

\section{Microsoft Kinect}
\label{sec:Microsoft Kinect Implementation}

\section{Robot}
\label{sec:Robot}
This section will describe arduino code for the robot. This will include how the arduino connects to the Kinect and the robots behaviour and calculations done by the arduino. If there was any comments in the original code, they have been removed in the listings for this chapter. \newline
A problem occurred when both the WiFi shield and the motor shield should work together. When the WiFi shield is trying to connect to the network and if any of the motor shield components are initialized at the beginning of the program, it isn't possible to connect to the network. If it is connected to the network before the motor shield components are initialized, it is possible to receive data and use the motor shield components.

The arduino setup function, is a function run once at the beginning of the running program, to setup the program as the name implies. In listing \ref{lst:setup} the setup function is shown, which in this project will connect to the private network, using the guide from arduinos own website \citep{wg}. \newline
Basically it will try to connect to the specific network by passing the network name and password to the function WiFi.begin(), given by the WiFi library. If the arduino can't connect to the network "Couldn't get a wifi connection" will be printed to the console and it will exit the program with the while(true) loop. Else the arduino will call the function server.begin() and is now connected to the network.

\begin{lstlisting}[caption={The arduino setup function}, label={lst:setup}]
void setup() {
	Serial.begin(9600);           
	Serial.println("Attempting to connect to WPA network...");
	status = WiFi.begin(ssid, pass);

	if ( status != WL\_CONNECTED) { 
		Serial.println("Couldn't get a wifi connection");
		while(true);
	} 
	else {
		server.begin();
		Serial.println("Connected to network");
	}	
	ip = WiFi.localIP();
	Serial.println(ip);
}
\end{lstlisting}

The arduino loop function is where the arduino's behaviour is programmed, since it will loop over the code in this function. In listing \ref{lst:loop} lines 2-15, the arduino is to receive data from the Kinect. The boolean \textbf{shouldDrive} is initialized to false, therefore it will enter the first if-statement. The client will be initialized to server.available(), which will make the arduino ready to receive data. The if-statement at line 4 it will receive the data from the Kinect and append the chars to the string \textbf{dataString}, which will contains the coordinate set for the objects impact position. \newline
At the if-statement at line 10, it will get the goalX and goalY coordinates from the string \textbf{dataString} and convert them to integers. Then setting the boolean \textbf{shouldDrive} = true, and set the string \textbf{dataString} = "". \newline
Since \textbf{shouldDrive} is now true, it will enter the else-statement at line 16. Lines 17-33 are setting up the arduino to get ready to drive to the received coordinate point. First it will initialize the motors, setting the attaching the interrupts for the motor encoders, setting the motorspeed and releaseing the motors so it won't drive. Also it will initialized two new integers \textbf{leftTemp} and \textbf{rightTemp} to the value of the integers of \textbf{leftTotal} and \textbf{rightTotal} at the given time in the program.

\begin{lstlisting}[caption={The arduino loop function}, label={lst:loop}]
void loop() {
	if (!shouldDrive) {
		client = server.available();
		if (client.connected()){
			while (client.available()) {
			char c = client.read(); 
			dataString += c;
			}
		}
		if (dataString != "") {
			goalX = dataString.substring(0,dataString.indexOf(';')).toInt();
			goalY = dataString.substring(dataString.indexOf(';')+1).toInt();
			shouldDrive = true;
			dataString = "";
		}
	} else {
		AF_DCMotor motorLeft(1);
		AF_DCMotor motorRight(2);
		pinMode(LEFTENCODERPIN, INPUT);
		attachInterrupt(3, incrementLeft, CHANGE);
		pinMode(RIGHTENCODERPIN, INPUT);
		attachInterrupt(2, incrementRight, CHANGE);
		
		motorLeft.setSpeed(255);
		motorRight.setSpeed(255);			
		motorLeft.run(RELEASE);
		motorLeftRun = false;
		motorRight.run(RELEASE);
		motorRightRun = false;
		
		leftTemp = leftTotal;
		rightTemp = rightTotal;
		
		if(posX <= goalX + margin && posX >= goalX - margin && posY <= goalY + margin && posY >= goalY - margin){
			stopIfAtGoal();
		} else {
			driveTowardsGoal();  
		}
		delay(10);
		updatePosAndHead();
	}
}
\end{lstlisting}



\section{Scheduling}
\label{sec:Scheduling implementation}
To calculate the worst-case execution time(WCET) for the systems tasks, several methods were tried out, which is described in Appendix \ref{chap:Increment three} in section \ref{sec:i3Scheduling}. 
The method used in this project to get the WCET for the system tasks, is to use the micros() function in the arduino IDE. \newline
Each task running time is measured with the micros() function, by running the function several time at using the highest time value as WCET. This was done for the functions updatePosAndHead(), driveTowardsGoal() and for when the Kinect program sends data till the arduino have received it via WiFi.

The results of the WCET of the tasks using the micros() function:
\begin{itemize}
	\item updatePosAndHead = 1076 microseconds
	\item driveTowardsGoal = 732 microseconds
	\item WiFi = 8433 microseconds
\end{itemize}
These results was used to make a schedulability analysis in the UPPAAL program, which will be described in the following section.
