\chapter{First increment}
\label{chap:First Increment}
Test om: Bounce, predefined area
Skal komme væk fra accelerotmeter og gyroscope. 

\section{Requirements}
\label{sec:i1Requirements}
The requirements which will be considered in this increment is mark with a blue colour.

\begin{itemize}
\item The trash bin should catch the trash if the user throws it towards the trash bin and within a predefined area
\begin{itemize}
\item \textcolor{blue}{The robots predefined area should be calculated from the hardware limitations of the motors’ speed}
\end{itemize}
\item The robot should know where it is positioned
\begin{itemize}
\item \textcolor{blue}{The robot should have a starting position, from where it should be able to calculate it's current position}
\end{itemize}
\item The robot should be able to detect and track the thrown trash
\begin{itemize}
\item \textcolor{blue}{The thrown trash should be detected and tracked by a Microsoft Kinect}
\item The Kinect should send the coordinates of the impact point of the trash to the robot
\end{itemize}
\item The robot should know where the the thrown trash will land
\begin{itemize}
\item Trajectory prediction should be used to calculate impact point of the thrown trash
%\item \textcolor{red}{The trajectory prediction should include a bounce prediction}
%\item \textcolor{red}{The bounce-shot should be thrown from a designated side of the camera}
%\item \textcolor{red}{The bounce-shot should bounce into the predefined area where the robot should catch the trash within }
\end{itemize}
\item The robot should be able to move the trash bin, such that the thrown trash lands inside the bin
\begin{itemize}
\item \textcolor{blue}{The robot should be able to turn, drive forward and drive backwards with a certain speed}
\item Multiple trash thrown should not be considered
\end{itemize}
\end{itemize}


\section{System design}
\label{sec:i1System Design}

\section{Implementation}
\label{sec:i1Implementation}

\section{Evaluation}
\label{sec:i1Evaluation}
