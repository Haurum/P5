\chapter{Theory}
\label{chap:Theory}

\section{Field of view}
\label{sec:Field of view}
Skal vi have noget om det synsfelt vi har til at opfange bolden i??

\section{Throwing}
\label{sec:ThrowingTheory}
When throwing the ball towards the trash bin, the group have two main methods in order to ensure that the sensor gets the most reliable data.

The first one being, that when throwing the ball, one must be standing with the camera/sensor to his side. From testing the Microsoft Kinect this has provided the best test results, when measuring for the distance of throw and comparing to the distance the sensor has given us.

The second method, is when throwing the ball, the trajectory curve of the ball, should have a high height so that it will cover more distance. In the future, this should not be a requirement either, but as of now, it should in order to give us more time to track, calculate and catch the object. 


\section{Detecting and tracking}
\label{sec:Detecting and trackingTheory}
The way the trash bin detects and track an object, happens when the object is of a defined form, such as a tennis ball, which we have chosen for this project. By predefining the object the sensors should look for, the error margin becomes smaller, and therefore this is how we plan on doing it, for this project. For the future, this point should not be predefined, as trash can be in various forms. 

In order to ensure that the sensor detect where the object is, the object has to come from a specific angel, that provides the sensor with the best actual distance. Once again, this is just for the project as of now, and in the future, this should not affect the way detecting works, as trash can be thrown from multiple angels, depending on the location of the person and the trash bin.


\section{Catching}
\label{sec:catchingTheory}
Catching concerns estimating where the thrown object will fall, and then moving the robot to that location before the objects lands to successfully catch the thrown object. In order to achieve the aforementioned objective, the following steps must be completed:

Tracking the projectile trajectory to estimate a location to catch or intercept the thrown object. The tracking is a continuous process and the the estimated location is transmitted to the robot repeatedly in intervals in the order of milliseconds. As the estimated location is transmitted to the robot, it will move accordingly to its relative position to the estimated location, this assumes some form coordinating system to enable the robot to move accurately. This could include first estimating whether or not the robot will even be able to reach the location before the object lands, or just move towards it regardless. It could also consider its own dimensions, as the bucket it will use to catch the robot puts restraints on how low a position the object can be, before the robot must in place to catch it, otherwise it will collide with the side of the bucket. 

(As a note, could include a formula of time it takes to turn, and its forward speed.)

Moving to the location includes time to turn the robot, and the time for its wheels to power it towards the location. As the robot will continuously receive input as to the location of the thrown object will present considerations as to which degree of precision is required and to prevent jittering the robot as it makes course corrections within very small intervals. For instance if the wheels take a proportionally large amount of time to turn, and the input received informs the robot to turn every 30 ms just 1 degree which is equal to pi divided by 180 in radians, would substantially slow the robot, depending on the size of the bucket and the precision of the estimation some imprecision can perhaps be afforded. 

\section{Trajectory prediction}
\label{sec:Trajectory prediction}
When the trash, referred to in this section as the projectile, is detected and the tracking of that projectile has started, the trajectory can be predicted. This prediction is limited to the amount of data sent by the sensory camera, meaning that for every camera reading, one detection of the object is gained. The precision of the prediction will increase according to the time a projectile has been tracked. \newline 
In this project, since the prediction is done indoor, the outdoor weather conditions that might affect the projectile trajectory is not considered. As well, the effects of air resistance, also called drag, will not be considered.

The trajectory of a projectile is the path of a thrown projectile without propulsion, affected by gravity. For calculating a trajectory of a projectile, the initial height, the angle which the projectile is launced from, the speed of the projectile at launch and the gravitational acceleration must be taken into account. \newline 
The initial height in this project is the height at which the projectile is detected, and the angle and speed at which the projectile is launched will be calculated from the first few trackings after detecting the projectile. The gravitational acceleration is considered as \(9.81m/s^2\), which is the standard near the earths surface. \newline
In this project we are interested in catching the projectile, and to do that, we need to calculate the distance the projectile travels before hitting the ground, and the amount of time before the projectile hits the ground. This is done with two mathematical formulas: \newline
\newline 
\begin{math}
g: \ the \ gravitational \ acceleration \ (9.81m/s^2)\newline
\theta: \ the \ angle \ at \ launch 
v: \ the \ speed \ at \ launch\newline
y_{0}: \ the \ initial \ height\newline
d: \ the \ total \ horizontal \ distance \ traveled \ \newline
t: \ the \ time \ of \ flight\newline
v_{vert}: \ the \ vertical \ velocity\newline
v_{hori}: \ the \ horizontal \ velocity\newline
t_{h}: \ the \ time \ since \ first \ detection\newline
d_{t}: \ the \ distance \ at \ time \ t\newline
\end{math}

From these variables we can express the formulas needed to provide the total distance traveled by the projectile, and the amount of time this would take.\newline
The distance traveled is:
\[d = \dfrac{v \ cos \ \theta}{g}(v \ sin \ \theta + \sqrt{(v \ sin \ \theta)^2 \ + \ 2gy_{0}})\] \newline
The time of flight is:
\[t = \dfrac{d}{v \ cos \ \theta} = \dfrac{v \ sin \ \theta \ + \ \sqrt{(v \ sin \ \theta)^2 \ + \ 2gy_{0}}}{g}\]
\newline

To calculate a trajectory of the projectile, the altitude and distance of the projectile at any time during the flight must be calculated, according to the initial height ( \(y_{0}\) ):
\[y = v_{vert}t - \dfrac{1}{2} gt^2\]
\[d_{t} = v_{hori}t\]

After this section, the projectile can be tracked, and the path for a given projectile can be calculated to a certain degree of correctness. 