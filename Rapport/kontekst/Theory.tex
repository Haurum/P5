\chapter{Theory}
\label{chap:Theory}

\section{Field of view}
\label{sec:Field of view}
Through out the report there will be mentioned the field of view and the predefined area. The field of view is the area where the sensor can detect the thrown objects, which is delimited by the sight of the sensor. \newline
The robots predefined area is an area marked uo with tape on the floor. Within this area is where the robot should catch the thrown object, which is detected by the sensor. Outside the predefined area, the robot have a specific starting position, from where it will start at for each throw. 

\section{Throwing}
\label{sec:ThrowingTheory}
When throwing the ball towards the trash bin, the group have two main methods in order to ensure that the sensor gets the most reliable data.

The first one being, that when throwing the ball, one must be standing with the camera/sensor to his side. From testing the Microsoft Kinect this has provided the best test results, when measuring for the distance of throw and comparing to the distance the sensor has given us.

The second method, is when throwing the ball, the trajectory curve of the ball, should have a high height so that it will cover more distance. In the future, this should not be a requirement either, but as of now, it should in order to give us more time to track, calculate and catch the object. 


\section{Detecting and tracking}
\label{sec:Detecting and trackingTheory}
The way the trash bin detects and track an object, happens when the object is of a defined form, such as a tennis ball, which we have chosen for this project. By predefining the object the sensors should look for, the error margin becomes smaller, and therefore this is how we plan on doing it, for this project. For the future, this point should not be predefined, as trash can be in various forms. 

In order to ensure that the sensor detect where the object is, the object has to come from a specific angel, that provides the sensor with the best actual distance. Once again, this is just for the project as of now, and in the future, this should not affect the way detecting works, as trash can be thrown from multiple angels, depending on the location of the person and the trash bin.


\section{Catching}
\label{sec:catchingTheory}

\section{Trajectory prediction}
\label{sec:Trajectory prediction}
When the trash, referred to in this section as the projectile, is detected and the tracking of that projectile has started, the trajectory can be predicted. This prediction is limited to the amount of data sent by the sensory camera, meaning that for every camera reading, one detection of the object is gained. The precision of the prediction will increase according to the time a projectile has been tracked. \newline 
In this project, since the prediction is done indoor, the outdoor weather conditions that might affect the projectile trajectory is not considered. As well, the effects of air resistance, also called drag, will not be considered.

The trajectory of a projectile is the path of a thrown projectile without propulsion, affected by gravity. For calculating a trajectory of a projectile, the initial height, the angle which the projectile is launced from, the speed of the projectile at launch and the gravitational acceleration must be taken into account. \newline 
The initial height in this project is the height at which the projectile is detected, and the angle and speed at which the projectile is launched will be calculated from the first few trackings after detecting the projectile. The gravitational acceleration is considered as \(9.81m/s^2\), which is the standard near the earths surface. \newline
In this project we are interested in catching the projectile, and to do that, we need to calculate the distance the projectile travels before hitting the ground, and the amount of time before the projectile hits the ground. This is done with two mathematical formulas: \newline
\newline 
\begin{math}
g: \ the \ gravitational \ acceleration \ (9.81m/s^2)\newline
\theta: \ the \ angle \ at \ launch 
v: \ the \ speed \ at \ launch\newline
y_{0}: \ the \ initial \ height\newline
d: \ the \ total \ horizontal \ distance \ traveled \ \newline
t: \ the \ time \ of \ flight\newline
v_{vert}: \ the \ vertical \ velocity\newline
v_{hori}: \ the \ horizontal \ velocity\newline
t_{h}: \ the \ time \ since \ first \ detection\newline
d_{t}: \ the \ distance \ at \ time \ t\newline
\end{math}

From these variables we can express the formulas needed to provide the total distance traveled by the projectile, and the amount of time this would take.\newline
The distance traveled is:
\[d = \dfrac{v \ cos \ \theta}{g}(v \ sin \ \theta + \sqrt{(v \ sin \ \theta)^2 \ + \ 2gy_{0}})\] \newline
The time of flight is:
\[t = \dfrac{d}{v \ cos \ \theta} = \dfrac{v \ sin \ \theta \ + \ \sqrt{(v \ sin \ \theta)^2 \ + \ 2gy_{0}}}{g}\]
\newline

To calculate a trajectory of the projectile, the altitude and distance of the projectile at any time during the flight must be calculated, according to the initial height ( \(y_{0}\) ):
\[y = v_{vert}t - \dfrac{1}{2} gt^2\]
\[d_{t} = v_{hori}t\]

After this section, the projectile can be tracked, and the path for a given projectile can be calculated to a certain degree of correctness. 