\chapter{Theory}
\label{chap:Theory}

\section{Field of view}
\label{sec:Field of view}
Through out the report there will be mentioned the field of view and the predefined area. The field of view is the area where the sensor can detect the thrown objects, which is delimited by the sight of the sensor. \newline
The robots predefined area is an area marked uo with tape on the floor. Within this area is where the robot should catch the thrown object, which is detected by the sensor. Outside the predefined area, the robot have a specific starting position, from where it will start at for each throw. 

\section{Throwing}
\label{sec:ThrowingTheory}
When throwing the ball towards the trash bin, the group have two main methods in order to ensure that the sensor gets the most reliable data.

The first one being, that when throwing the ball, one must be standing with the camera/sensor to his side. From testing the Microsoft Kinect this has provided the best test results, when measuring for the distance of throw and comparing to the distance the sensor has given us.

The second method, is when throwing the ball, the trajectory curve of the ball, should have a high height so that it will cover more distance. In the future, this should not be a requirement either, but as of now, it should in order to give us more time to track, calculate and catch the object. 


\section{Detecting and tracking}
\label{sec:Detecting and trackingTheory}
The way the trash bin detects and track an object, happens when the object is of a defined form, such as a table tennis ball, which we have chosen for this project. By predefining the object the sensors should look for, the error margin becomes smaller. % For the future, this point should not be predefined, as trash can be in various forms. 

In order to ensure that the sensor detect where the object is, the object has to come from a specific side of the Kinect, that provides the sensor with the best actual distance. Once again, this is just for the project as of now, and in the future, this should not affect the way detecting works, as trash can be thrown from multiple angels, depending on the location of the person and the trash bin.


\section{Catching}
\label{sec:catchingTheory}
Catching concerns estimating where the thrown object will fall, and then moving the robot to that location before the objects lands to successfully catch the thrown object. In order to achieve the aforementioned objective, the following steps must be completed:

Tracking the projectile trajectory to estimate a location to catch or intercept the thrown object. The tracking is a continuous process and the estimated location is transmitted to the robot repeatedly in intervals in the order of milliseconds. As the estimated location is transmitted to the robot, it will move accordingly to its relative position to the estimated location, this assumes some form coordinating system to enable the robot to move accurately. This could include first estimating whether or not the robot will even be able to reach the location before the object lands, or just move towards it regardless. It could also consider its own dimensions, as the bucket it will use to catch the robot puts restraints on how low a position the object can be, before the robot must in place to catch it, otherwise it will collide with the side of the bucket. 

(As a note, could include a formula of time it takes to turn, and its forward speed.)

Moving to the location includes time to turn the robot, and the time for its wheels to power it towards the location. As the robot will continuously receive input as to the location of the thrown object will present considerations as to which degree of precision is required and to prevent jittering the robot as it makes course corrections within very small intervals. For instance if the wheels take a proportionally large amount of time to turn, and the input received informs the robot to turn every 30 ms just 1 degree which is equal to pi divided by 180 in radians, would substantially slow the robot, depending on the size of the bucket and the precision of the estimation some imprecision can perhaps be afforded. 
