\chapter{Increment two}
\label{chap:Increment two}
This chapter is just like the chapter above. The results from increment one will be discussed and implemented and this should result in the same, new or changed requirements as well.  

\section{Requirements}
\label{sec:i2Requirements}
The requirement marked with blue will be considered throughout this chapter.

\begin{itemize}
	\item The trash bin should catch the trash if the user throws it towards the trash bin and within a predefined area
	\begin{itemize}
		\item {The robots predefined area should be calculated from the hardware limitations of the motors’ speed}
	\end{itemize}
	\item The robot should know where it is positioned
	\begin{itemize}
		\item \textcolor{blue}{The robot should have a starting position, from where it should be able to calculate it's current position through calculations of the motor encoders}
		\item {The robot's starting point should be placed outside its predefined area, such that it moves forward into the area}
	\end{itemize}
	\item The robot should be able to detect and track the thrown trash
	\begin{itemize}
		\item {The thrown trash should be detected and tracked by a Microsoft Kinect}
		\item \textcolor{blue}{The Kinect should send the coordinates of the impact point of the trash to the robot}
	\end{itemize}
	\item The robot should know where the the thrown trash will land
	\begin{itemize}
		\item {Trajectory prediction should be used to calculate impact point of the thrown trash}
	\end{itemize}
	\item The robot should be able to move the trash bin, such that the thrown trash lands inside the bin
	\begin{itemize}
		\item {The robot should be able to turn and drive forward}
	\end{itemize}
	\item \textcolor{blue}{The robot should be able to receive data from a computer, through a wireless network}
\end{itemize}

\section{System design}
 \label{sec:i2System design} 
In this section, the marked requirements should be fulfilled and will be described and explained.  

\subsection{Wifi Shield}
\label{sec:Wifi Shield SD}
When the Kinect have found the coordinates for the collision point of the thrown object, these coordinates should be sent to the robot, for it to be able to move to the collision point and catch the object. 
The data should be sent, as earlier mentioned in section \ref{sec: Arduino Wifi Shield}, by the Arduino wifi shield. The Arduino and the computer running the Kinect program, should be connected to their own network, with the computer sending data to the wifi shields IP-address.

\subsection{Robot positioning}
\label{sec:Robot positioning System Design}
In the previous chapter, it was decided to use the motor's encoders to calculate the heading of the robot and the distance it has moved. The encoders will keep track of the number of degrees the specific motor have turned, which can be calculated to a distance using the wheels circumference.
The coordinate set for the collision point of the object and the robots current position, will be used to calculate the heading, which the robot should have to reach the collision point. 

(Hvordan vi oversætter Kinect X/Y til robot X/Y)

\section{Implementation}
\label{sec:i2Implementation}
This section explains how the marked requirements were actually implemented in the project and whether or not the requirements have been changed, removed or been fulfilled. 

\subsection{Wifi Shield}
\label{sec:Wifi Shield Implementation}
To implement the Wifi Shield as part of the Arduino, the router connecting the Wifi Shield with a computer had to be port forwarded. The private network hosted by the D-link router had a specific ssid and password, which had to be explicitly assigned in the Arduino code for the Wifi Shield. The setup in the Arduino code was used to start the connection between the computer and the Wifi Shield, as well as printing the log of the connection. The setup for the wifi shield can be found in listing \ref{ws}.

\begin{lstlisting}[caption={Connecting the Wifi shield to the network}, label={ws}]
void setup() {
  Serial.begin(9600);

  Serial.println("Attempting to connect to WPA network...");
  status = WiFi.begin(ssid, pass);

  if ( status != WL_CONNECTED) { 
    Serial.println("Couldn't get a wifi connection");
    while(true);
  } 
  else {
    server.begin();
    Serial.println("Connected to network");
  }
  ip = WiFi.localIP();
  Serial.println(ip);
}
\end{lstlisting}

After connecting the Wifi Shield to the router, the Arduino is ready to receive data. The data will be sent from a laptop, which connects and sends data as shown in \ref{cws}. A tcpClient is created and connected to the IP of the Wifi Shield, and the port that has been forwarded. The data sent in the code is a string "test", which is sent as a byte-array, and after sending the data, the tcpClient closes the connection.

\begin{lstlisting}[caption={Connecting the computer to the Wifi Shield}, label={cws}]
tcpClient.Connect("192.168.0.100", 9999);
string data = "test";
const int IntSize = 4;
byte[] bytedata;
Stream stream = tcpClient.GetStream();

stream.Write(Encoding.UTF8.GetBytes(data.ToCharArray()), 0, data.Length);
tcpClient.Close();
\end{lstlisting}

When both the computer and the Wifi shield has made the connection, and the computer has sent the data, the Arduino can then receive the data. This is done through a Wifi client, from the Arduino Wifi library. When assigning the client to the server.available, the value of the client in an if-statement would be true if any client from the server has data available for reading, or false if no data was available. If a client is connected, and the client has data available for reading, the next byte will be appended on the string readString, and after all the data available has been read, the full string sent by the client will be printed.

\begin{lstlisting}[caption={Receiving data from the computer}, label={rdc}]
client = server.available();
while (client.connected()){
  if (client.available()) {
    char c = client.read(); 
    readString += c;
  }
}
if(readString != ""){
  Serial.println(readString);
  readString = "";
}
\end{lstlisting}

\subsection{Robot positioning}
\label{sec:Robot positioning Implementation}
In the top of the Arduino program, a lot of variables were declared, which can be seen in listing \ref{declarations}. The two volatile integers leftTotal and rightTotal are used to count the number of degrees the motor have turned. leftTotal and rightTotal are volatile integers because the value of the variables can be changed by other things than the code, which in this case is the motors. \newline
motorLeftRun and motorRighRun are booleans used to check whether the motors are running or not. \newline
The variable called heading have the value of the robot's heading, which in this project will be the degrees the robot's direction is pointing away from the y-axis of its predefined area. Then robot's current position will be saved as coordinates in the variables posX and posY. \newline
UNSIGNED INT TID - ER IKKE BRUGT \newline
Several integers are declared which will get further explanation later in the section. 
Also in the top of the program some identifiers have been defined and the necessary libraries have been included to the program.  

\begin{lstlisting}[caption={The defined and declared variables}, label={declarations}]
volatile int leftTotal = 0;
volatile int rightTotal = 0;
bool motorLeftRun = false;
bool motorRightRun = false;
double heading = 0.0, posX = 0.0, posY = 0.0;
unsigned int tid;
int leftTemp, rightTemp, goalX = 230, goalY = 330, margin = 10, loopcount = 0, signalCount = 0;
\end{lstlisting}

An arduino program is split up into a setup function and a loop function, where the code in the loop function should describe the robot's behaviour. The setup function are run once for every power up or reset of the Arduino board.
The setup function can be found in listing \ref{setup}. Two pins are setup as inputs for the arduino board and an interrupt is attached to each of them. Every time a change occurs for the input pins, which in this case is every time the motors have turned 1 degree, the functions incrementLeft and incrementRight are called accordingly. The two functions increments the counters leftTotal and rightTotal which are counting the number of degrees the motors have turned. \newline
Next up the motors are turned on. This is done by setting their speed, which here is set to the highest speed possible, and then both motors are ret to RELEASE, which will stop the motors. Also the booleans motorLeftRun and motorRightRun are set to false. \newline
At the bottom of the setup funtion, then leftTotal's and the rightTotal's values have been assigned to the temporary variables leftTemp and rightTemp.

\begin{lstlisting}[caption={The setup function}, label={setup}]'
pinMode(LEFTENCODERPIN, INPUT);
attachInterrupt(LEFTPINTOINTERRUPT, incrementLeft, CHANGE);
pinMode(RIGHTENCODERPIN, INPUT);
attachInterrupt(RIGHTPINTOINTERRUPT, incrementRight, CHANGE);

motorLeft.setSpeed(255);
motorRight.setSpeed(255);

motorLeft.run(RELEASE);
motorLeftRun = false;
motorRight.run(RELEASE);
motorRightRun = false;

leftTemp = leftTotal;
rightTemp = rightTotal;
\end{lstlisting}

\section{Evaluation}
\label{sec:i2Evaluation}
This section will include a short summary of the chapter and conclude with an updated requirement list, showing changes from the first requirement list in the chapter. 