\chapter{Design specification}
\label{chap:Design specification}
This chapter will present the requirements and the way they were conducted for the project, the requirements will be examined at the end of the report to see if they have been fulfilled. The last section in the chapter will describe the delimitations made for the project.

\section{System requirements}
\label{sec:System requirements}
The requirements for the project have been assembled through out several chapters. First in the Analysis chapter \ref{chap:Analysis}, the general requirements were set. Then in the Hardware chapter \ref{chap:Hardware}, the hardware would set some limitations for the project which lead to some more detailed requirements in the form of subrequirements.
Through out the work with the robot problems occurred and the requirements therefore changed. The changes done to the requirements though out the project, can be seen in Appendix (Referer til appendix A her) og (Referer til appendix B her).

The requirements for the project have been conducted thorugh out the whole process and changes were made accordingly, to adapt to the hardware's limations and the problems found. The requirements can be seen in the following paragraphs. 

\begin{itemize}
	\item The trash bin should catch the trash if the user throws it towards the trash bin and within a predefined area
	\begin{itemize}
		\item {The robots predefined area should be calculated from the hardware limitations of the motors’ speed}
	\end{itemize}
	\item The robot should know where it is positioned
	\begin{itemize}
		\item{The robot should have a starting position, from where it should be able to calculate it's current position through calculations of the motory encoders}
		\item {The robot's starting point should be placed outside its predefined area, such that it moves forward into the area}
	\end{itemize}
	\item The robot should be able to detect and track the thrown trash
	\begin{itemize}
		\item {The thrown trash should be detected and tracked by a Microsoft Kinect}
		\item {The Kinect should send the coordinates of the impact point of the trash to the robot}
	\end{itemize}
	\item The robot should know where the the thrown trash will land
	\begin{itemize}
		\item {Trajectory prediction should be used to calculate impact point of the thrown trash}
	\end{itemize}
	\item The robot should be able to move the trash bin, such that the thrown trash lands inside the bin
	\begin{itemize}
		\item {The robot should be able to turn and drive forward}
	\end{itemize}
	\item {The robot should be able to receive data from a computer, through a wireless network}
\end{itemize}

\section{Delimitations}
\label{sec:Delimitations}
The group have made two delimitations for the project. First the group have decided that multiple object thrown should not be considered, so only one object should be thrown at a time. \newline
It have also been decided that the robot should not drive backwards. The robot starts outside its predefined area and will only drive forward into that. This means that if the robot is already heading towards a point in its predefined area and the Kinect sends a new point, which is behind the robot, it will not be able to drive to the new point given by the Kinect. 