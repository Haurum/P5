\chapter{Toolchain}
The following tools has been used to create the report and the product of the project. Every tool in the toolchain has a description stating the benefits and use of the tool.


\begin{itemize}
	\item GitHub
	\item TeX Studio
	\item Visual Studio
	\item Arduino IDE 1.0.3
	\item BoundT
	\item UPPAAL
	\item ArduinoUnit
\end{itemize}


\section*{Use of tools}
\textbf{GitHub} is used with the intention of merging and sharing documents. This tool is also used here to verify each version of the documentation and code, to ensure that everyone has the latest update of the project.\newline
\textbf{TeX Studio} is used as a writing environment when writing with the markup language LaTeX. After some experience it has proven to be a great tool for creating a report. This tool in concoction with GitHub makes it possible to work on the project in pairs or alone, with little merging conflicts. \newline
\textbf{Visual Studio} is used in this project to create the C\# code for the Kinect. \newline
\textbf{Arduino IDE 1.0.3} is used in this project to code the Arduino. Any later IDE is not suitable for this project, which will be explained in the report. \newline
\textbf{BoundT} is used to calculate the bounds of the code, which will be used in a WCET-analysis. \citep{boundt} \newline
\textbf{UPPAAL} is an integrated tool environment for modeling, validation and verification of real-time system. \citep{uppaal} \newline
\textbf{ArduinoUnit} is a unit test framework for arduino projects and is used for the unit tests in the tests chapter, \ref{chap:Tests}, of this project. \citep{au}


